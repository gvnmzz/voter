
%% bare_jrnl.tex
%% V1.4a
%% 2014/09/17
%% by Michael Shell
%% see http://www.michaelshell.org/
%% for current contact information.
%%
%% This is a skeleton file demonstrating the use of IEEEtran.cls
%% (requires IEEEtran.cls version 1.8a or later) with an IEEE
%% journal paper.
%%
%% Support sites:
%% http://www.michaelshell.org/tex/ieeetran/
%% http://www.ctan.org/tex-archive/macros/latex/contrib/IEEEtran/
%% and
%% http://www.ieee.org/

%%*************************************************************************
%% Legal Notice:
%% This code is offered as-is without any warranty either expressed or
%% implied; without even the implied warranty of MERCHANTABILITY or
%% FITNESS FOR A PARTICULAR PURPOSE! 
%% User assumes all risk.
%% In no event shall IEEE or any contributor to this code be liable for
%% any damages or losses, including, but not limited to, incidental,
%% consequential, or any other damages, resulting from the use or misuse
%% of any information contained here.
%%
%% All comments are the opinions of their respective authors and are not
%% necessarily endorsed by the IEEE.
%%
%% This work is distributed under the LaTeX Project Public License (LPPL)
%% ( http://www.latex-project.org/ ) version 1.3, and may be freely used,
%% distributed and modified. A copy of the LPPL, version 1.3, is included
%% in the base LaTeX documentation of all distributions of LaTeX released
%% 2003/12/01 or later.
%% Retain all contribution notices and credits.
%% ** Modified files should be clearly indicated as such, including  **
%% ** renaming them and changing author support contact information. **
%%
%% File list of work: IEEEtran.cls, IEEEtran_HOWTO.pdf, bare_adv.tex,
%%                    bare_conf.tex, bare_jrnl.tex, bare_conf_compsoc.tex,
%%                    bare_jrnl_compsoc.tex, bare_jrnl_transmag.tex
%%*************************************************************************


% *** Authors should verify (and, if needed, correct) their LaTeX system  ***
% *** with the testflow diagnostic prior to trusting their LaTeX platform ***
% *** with production work. IEEE's font choices and paper sizes can       ***
% *** trigger bugs that do not appear when using other class files.       ***                          ***
% The testflow support page is at:
% http://www.michaelshell.org/tex/testflow/



\documentclass[journal]{IEEEtran}

\usepackage[T1]{fontenc}
\usepackage[english]{babel}
\usepackage{amsmath}
\usepackage{amsfonts}
\usepackage{amssymb}
\usepackage{graphicx}
\usepackage{parskip}
\usepackage{lmodern}
\usepackage{braket}
\usepackage{mathtools}
\usepackage{caption}
\usepackage{subcaption}
\usepackage{float}
\usepackage{bm}
\usepackage{booktabs}
\usepackage{scrextend}

\usepackage{color}
\usepackage{fancyvrb}
\usepackage{listings}
\usepackage{bm}
\usepackage[hidelinks]{hyperref}


% correct bad hyphenation here
\hyphenation{op-tical net-works semi-conduc-tor}


\begin{document}
%
% paper title
% Titles are generally capitalized except for words such as a, an, and, as,
% at, but, by, for, in, nor, of, on, or, the, to and up, which are usually
% not capitalized unless they are the first or last word of the title.
% Linebreaks \\ can be used within to get better formatting as desired.
% Do not put math or special symbols in the title.
\title{Complexity in the electoral competition game}
%
%
% author names and IEEE memberships
% note positions of commas and nonbreaking spaces ( ~ ) LaTeX will not break
% a structure at a ~ so this keeps an author's name from being broken across
% two lines.
% use \thanks{} to gain access to the first footnote area
% a separate \thanks must be used for each paragraph as LaTeX2e's \thanks
% was not built to handle multiple paragraphs
%

\author{Giovanni Mizzi,~\IEEEmembership{ID 1450441}}% <-this % stops a space
\thanks{G. Mizzi is with the Center for Complexity Science, Zeeman Building, University of Warwickm, CV4 7AL, email: g.mizzi@warwick.ac.uk.}% <-this % stops a space
\thanks{This is a technical report for the module "Complexity in Social Sciences" IM903.}% <-this % stops a space


% note the % following the last \IEEEmembership and also \thanks - 
% these prevent an unwanted space from occurring between the last author name
% and the end of the author line. i.e., if you had this:
% 
% \author{....lastname \thanks{...} \thanks{...} }
%                     ^------------^------------^----Do not want these spaces!
%
% a space would be appended to the last name and could cause every name on that
% line to be shifted left slightly. This is one of those "LaTeX things". For
% instance, "\textbf{A} \textbf{B}" will typeset as "A B" not "AB". To get
% "AB" then you have to do: "\textbf{A}\textbf{B}"
% \thanks is no different in this regard, so shield the last } of each \thanks
% that ends a line with a % and do not let a space in before the next \thanks.
% Spaces after \IEEEmembership other than the last one are OK (and needed) as
% you are supposed to have spaces between the names. For what it is worth,
% this is a minor point as most people would not even notice if the said evil
% space somehow managed to creep in.



% The paper headers
\markboth{Computational Social Systems,~Vol.~13, No.~9, May~2015}%
{Mizzi \MakeLowercase{\textit{et al.}}: Complexity in the electoral competition game}
% The only time the second header will appear is for the odd numbered pages
% after the title page when using the twoside option.
% 
% *** Note that you probably will NOT want to include the author's ***
% *** name in the headers of peer review papers.                   ***
% You can use \ifCLASSOPTIONpeerreview for conditional compilation here if
% you desire.




% If you want to put a publisher's ID mark on the page you can do it like
% this:
%\IEEEpubid{0000--0000/00\$00.00~\copyright~2014 IEEE}
% Remember, if you use this you must call \IEEEpubidadjcol in the second
% column for its text to clear the IEEEpubid mark.



% use for special paper notices
%\IEEEspecialpapernotice{(Invited Paper)}




% make the title area
\maketitle

% As a general rule, do not put math, special symbols or citations
% in the abstract or keywords.
\begin{abstract}
The abstract goes here.
\end{abstract}

% Note that keywords are not normally used for peerreview papers.
\begin{IEEEkeywords}
Voter model, Game Theory, Complexity.
\end{IEEEkeywords}


% For peer review papers, you can put extra information on the cover
% page as needed:
% \ifCLASSOPTIONpeerreview
% \begin{center} \bfseries EDICS Category: 3-BBND \end{center}
% \fi
%
% For peerreview papers, this IEEEtran command inserts a page break and
% creates the second title. It will be ignored for other modes.
\IEEEpeerreviewmaketitle



\section{Introduction}
% The very first letter is a 2 line initial drop letter followed
% by the rest of the first word in caps.
% 
% form to use if the first word consists of a single letter:
% \IEEEPARstart{A}{demo} file is ....
% 
% form to use if you need the single drop letter followed by
% normal text (unknown if ever used by IEEE):
% \IEEEPARstart{A}{}demo file is ....
% 
% Some journals put the first two words in caps:
% \IEEEPARstart{T}{his demo} file is ....
% 
% Here we have the typical use of a "T" for an initial drop letter
% and "HIS" in caps to complete the first word.
\IEEEPARstart{E}{lectoral} competitions have always been fascinating for Social Scientist. They represent the best way to take decisions the part of humanity who decided to govern itself with democracy could come up, and could be dated back to the ancient Greece. Since the outcome of an electoral competition is something that influences, on different levels, every part of the society, great effort has been put in trying to solve the problem of its prediction.
The traditional model used in Game Theory for this problem is the Hotelling's Game. This is a completely rational model and, starting from some simple assumptions, in a very simple way, it explains what should happen at the equilibrium. The assumptions it relies on are so simple, though, that they hardly capture the complexity of the real problem if one does not introduce some complication in it.
On the other hand, a very simple model that capture the complexity of the model is the Voter model, which is one of the simplest way to model the diffusion of opinions taking into account randomness. This model too, taken alone, cannot grasp in its simplicity lots of the dynamics of the real-world problem.
After illustrating the basic concept at the base of these two model, a combination of them will be considered, trying to put together the principles of Hotelling's Game with the complexity of the Voter model, to illustrate what are the difficulties in solving the very complex and complicated problem of predicting the result of electoral competition.

\section{Hotelling's Game}

In the simplest version of Hotelling's Game the players are two candidates that only care about winning and they can take a position concerning a certain policy. This position is modelled as a number in the interval $[0,1]$ where $0$ corresponds to the extreme left and $1$ to the extreme right. The distribution of the voters opinions is unknown, but it is assumed to be symmetric for simplicity. Every person will vote the candidate with the position closer to her own. In these conditions, it is intuitive to understand that the closer one candidate position is to the centre, i.e. closer to $0.5$, the greater is the number of people that will vote for her. 
The equilibrium strategy is to be in the centre, to vote the median position, the one that has half of the voters on her left and half on her right. In this case, both the candidates have a chance of $1/2$ to be elected. This is an equilibrium because none of them would like to deviate to a position different from the centre. In fact, if the Left, for instance, would deviate to the left, the Right candidate would take the votes of all the people on the right of the median plus the people that are closer than half way from her position to that of the other candidate, and this number is bigger than the half of the people.

The model is evidently very simple, and so is the result that it produces. Of course, there are many variants of the game, for example in which the players do not care only about winning but also care about their position, in which the distribution of the voters opinions is not symmetric, or in which there are more than two candidates. It is very easy to see that this model, already for the case of three candidates, does not have an equilibrium. In fact the Right and Left candidates will always tend to go towards the centre, while the Centre candidate will always tend to go past one of the other two candidates.

\section{The Voter model}
This model is very simple to describe, and also in this case a very simple rule is prescribed. Since the agents are the people this time, the dynamics are much more complex than in the previous case due to the presence of a degree of randomness. 

A social network is considered, in which the nodes are people and the presence of a link indicates that the connected people know each other. Every person in the network has an opinion about some topic which is dichotomic, it can be "yes" or "no", "true" or "false", etc. The model works in an algorithmic fashion, and at each time step a person is selected and it changes her opinion to that of one of her neighbours, chosen randomly.

This model, from a certain point of view is even simpler than the Hotelling's game, but it grasps the fact that people communicate with each other and opinions do spread in the network in a complex way that does not have a simple equilibrium, at least not a static one.
Of course this model can be complicated introducing, for instance, the possibility of opinions on different matters, and of different opinions on the same matter, also it is possible to introduce rules for conformity and homophily.

\section{The Model}
The model presented in the present work puts together elements both from the Hotelling's game of electoral competition and from the Voter model in an attempt to construct a still not-so-complicated model, that better grasps some features of the real problem of electoral competition.

First, let's consider a social network, as in the Voter model. All the models should be intended on a generic network, and since the problem concerns social interactions, it would probably be a scale-free network. In this work, though, a simple 2D lattice with periodic boundary conditions has been used for reasons of visualisation. 

The first difference that can be introduced in the Voter model to try and match it with the Hotelling's game is to let people take opinions in an interval $[0,1]$. This is already a very big difference and a complication, but it is intended to model the fact that the opinion of people are not always black or white, but more often they are shades of gray. This is also convenient, because it is exactly the way in which the positions of people are modelled in the Hotelling's game.

The next difference is in the rule with which the opinions spread. Keeping the rule of the Voter model does not seems so convenient with a continuum distribution of opinions. First, because in that case, our distribution would still be discrete, due to the finiteness of the network. There cannot be more than $N$ different opinions to start with, $N$ being the number of nodes, and the number of opinions can only decrease. Second, it may sound strange in this context the idea that talking with a person she is able to convince her interlocutor of exactly her opinion. A better way to model this interaction could be to let the person take an intermediate opinion between the two, in particular the arithmetic mean since the position are numbers.

Now let's put the two candidates in the game. They both have positions, and let's assume that one of them is more oriented to Left and one more oriented to Right. So the Left candidate's position will be a number in $[0,0.5]$ and the Right one's position will be in $[0.5,1]$, at random. 
It is possible to imagine that also their position play a role in people opinions. In particular, since people tend to agree more with people that have a position closer to them, it is possible to imagine that the candidate that would influence more one person opinion is the one with a position closer to her.
So at this point we can imagine that the person selected at a certain time step will shift her opinion to a weighted average of her opinion, the one of a neighbour and the one of the candidate closer to her opinion. A reasonable choice is that her opinions weights more than the one of the friend that weights more than the one of the candidate.
%%%WIRED SCIENZA SPIEGA

The last difference that one can think to introduce is a way for the opinion of people to influence the candidate position.
Starting from the assumption that the candidate only cares about winning (as in Hotelling's game), it is possible to think that through some mean, like surveys or online data such as Twitter, Facebook or Google, each candidate is able to know what is the distribution of opinions of people. Of course it is very unlikely for her to be able to sample the entire population, so let's assume she is able to sample only a small fraction of the population. Given the data available to her, she may decide to take the position of the median of the distribution she found if she only cares about winning. Otherwise, she could decide to take as her position the arithmetic mean of the median of the distribution and her position if for some reason she also cares about the policy.

These where simple

% An example of a floating figure using the graphicx package.
% Note that \label must occur AFTER (or within) \caption.
% For figures, \caption should occur after the \includegraphics.
% Note that IEEEtran v1.7 and later has special internal code that
% is designed to preserve the operation of \label within \caption
% even when the captionsoff option is in effect. However, because
% of issues like this, it may be the safest practice to put all your
% \label just after \caption rather than within \caption{}.
%
% Reminder: the "draftcls" or "draftclsnofoot", not "draft", class
% option should be used if it is desired that the figures are to be
% displayed while in draft mode.
%
%\begin{figure}[!t]
%\centering
%\includegraphics[width=2.5in]{myfigure}
% where an .eps filename suffix will be assumed under latex, 
% and a .pdf suffix will be assumed for pdflatex; or what has been declared
% via \DeclareGraphicsExtensions.
%\caption{Simulation results for the network.}
%\label{fig_sim}
%\end{figure}

% Note that IEEE typically puts floats only at the top, even when this
% results in a large percentage of a column being occupied by floats.


% An example of a double column floating figure using two subfigures.
% (The subfig.sty package must be loaded for this to work.)
% The subfigure \label commands are set within each subfloat command,
% and the \label for the overall figure must come after \caption.
% \hfil is used as a separator to get equal spacing.
% Watch out that the combined width of all the subfigures on a 
% line do not exceed the text width or a line break will occur.
%
%\begin{figure*}[!t]
%\centering
%\subfloat[Case I]{\includegraphics[width=2.5in]{box}%
%\label{fig_first_case}}
%\hfil
%\subfloat[Case II]{\includegraphics[width=2.5in]{box}%
%\label{fig_second_case}}
%\caption{Simulation results for the network.}
%\label{fig_sim}
%\end{figure*}
%
% Note that often IEEE papers with subfigures do not employ subfigure
% captions (using the optional argument to \subfloat[]), but instead will
% reference/describe all of them (a), (b), etc., within the main caption.
% Be aware that for subfig.sty to generate the (a), (b), etc., subfigure
% labels, the optional argument to \subfloat must be present. If a
% subcaption is not desired, just leave its contents blank,
% e.g., \subfloat[].


% An example of a floating table. Note that, for IEEE style tables, the
% \caption command should come BEFORE the table and, given that table
% captions serve much like titles, are usually capitalized except for words
% such as a, an, and, as, at, but, by, for, in, nor, of, on, or, the, to
% and up, which are usually not capitalized unless they are the first or
% last word of the caption. Table text will default to \footnotesize as
% IEEE normally uses this smaller font for tables.
% The \label must come after \caption as always.
%
%\begin{table}[!t]
%% increase table row spacing, adjust to taste
%\renewcommand{\arraystretch}{1.3}
% if using array.sty, it might be a good idea to tweak the value of
% \extrarowheight as needed to properly center the text within the cells
%\caption{An Example of a Table}
%\label{table_example}
%\centering
%% Some packages, such as MDW tools, offer better commands for making tables
%% than the plain LaTeX2e tabular which is used here.
%\begin{tabular}{|c||c|}
%\hline
%One & Two\\
%\hline
%Three & Four\\
%\hline
%\end{tabular}
%\end{table}


% Note that the IEEE does not put floats in the very first column
% - or typically anywhere on the first page for that matter. Also,
% in-text middle ("here") positioning is typically not used, but it
% is allowed and encouraged for Computer Society conferences (but
% not Computer Society journals). Most IEEE journals/conferences use
% top floats exclusively. 
% Note that, LaTeX2e, unlike IEEE journals/conferences, places
% footnotes above bottom floats. This can be corrected via the
% \fnbelowfloat command of the stfloats package.




\section{Conclusion}
The conclusion goes here.





% if have a single appendix:
%\appendix[Proof of the Zonklar Equations]
% or
%\appendix  % for no appendix heading
% do not use \section anymore after \appendix, only \section*
% is possibly needed

% use appendices with more than one appendix
% then use \section to start each appendix
% you must declare a \section before using any
% \subsection or using \label (\appendices by itself
% starts a section numbered zero.)
%


\appendices
\section{Proof of the First Zonklar Equation}
Appendix one text goes here.

% you can choose not to have a title for an appendix
% if you want by leaving the argument blank
\section{}
Appendix two text goes here.


% use section* for acknowledgment
\section*{Acknowledgment}


The authors would like to thank...


% Can use something like this to put references on a page
% by themselves when using endfloat and the captionsoff option.
\ifCLASSOPTIONcaptionsoff
  \newpage
\fi



% trigger a \newpage just before the given reference
% number - used to balance the columns on the last page
% adjust value as needed - may need to be readjusted if
% the document is modified later
%\IEEEtriggeratref{8}
% The "triggered" command can be changed if desired:
%\IEEEtriggercmd{\enlargethispage{-5in}}

% references section

% can use a bibliography generated by BibTeX as a .bbl file
% BibTeX documentation can be easily obtained at:
% http://www.ctan.org/tex-archive/biblio/bibtex/contrib/doc/
% The IEEEtran BibTeX style support page is at:
% http://www.michaelshell.org/tex/ieeetran/bibtex/
%\bibliographystyle{IEEEtran}
% argument is your BibTeX string definitions and bibliography database(s)
%\bibliography{IEEEabrv,../bib/paper}
%
% <OR> manually copy in the resultant .bbl file
% set second argument of \begin to the number of references
% (used to reserve space for the reference number labels box)
\begin{thebibliography}{1}

\bibitem{IEEEhowto:kopka}
H.~Kopka and P.~W. Daly, \emph{A Guide to \LaTeX}, 3rd~ed.\hskip 1em plus
  0.5em minus 0.4em\relax Harlow, England: Addison-Wesley, 1999.

\end{thebibliography}

% biography section
% 
% If you have an EPS/PDF photo (graphicx package needed) extra braces are
% needed around the contents of the optional argument to biography to prevent
% the LaTeX parser from getting confused when it sees the complicated
% \includegraphics command within an optional argument. (You could create
% your own custom macro containing the \includegraphics command to make things
% simpler here.)
%\begin{IEEEbiography}[{\includegraphics[width=1in,height=1.25in,clip,keepaspectratio]{mshell}}]{Michael Shell}
% or if you just want to reserve a space for a photo:

\begin{IEEEbiography}{Michael Shell}
Biography text here.
\end{IEEEbiography}

% if you will not have a photo at all:
\begin{IEEEbiographynophoto}{John Doe}
Biography text here.
\end{IEEEbiographynophoto}

% insert where needed to balance the two columns on the last page with
% biographies
%\newpage

\begin{IEEEbiographynophoto}{Jane Doe}
Biography text here.
\end{IEEEbiographynophoto}

% You can push biographies down or up by placing
% a \vfill before or after them. The appropriate
% use of \vfill depends on what kind of text is
% on the last page and whether or not the columns
% are being equalized.

%\vfill

% Can be used to pull up biographies so that the bottom of the last one
% is flush with the other column.
%\enlargethispage{-5in}



% that's all folks
\end{document}


